\section{Jacobians of graphs I (Lecture 5)}
Let $X$ be a smooth projective curve over $\mathbb{C}((t))$. Let $\mathcal{X}$ be a regular semistable snc-model and $G$ the dual graph of $\overline{\mathcal{X}} = \mathcal{X}_{\mathbb{C}}$.

The restriction map $\mathrm{Pic}(\mathcal{X}) \to \mathrm{Pic}(X)$ has a kernel generated by $\mathcal{O}(\overline{\mathcal{X}}_i)$, where $\overline{\mathcal{X}} = \overline{\mathcal{X}}_0 \cup \ldots \cup \overline{\mathcal{X}}_n$ with all $\overline{\mathcal{X}}_i$ irreducible.

Let $L$ be a line bundle on $X$. Let $\mathcal{L}$ be a model of $L$; then $\mathcal{L}$ is a line bundle on $\mathcal{X}$ and $\mathcal{L}|_X \cong L$.

Consider the matrix $\Delta$ such that $\Delta_{ij} = \mathrm{deg}(\mathcal{O}(\overline{\mathcal{X}}_i)|_{\overline{\mathcal{X}}_j})$. Note: if $i \neq j$ then $$\Delta_{ij} = \#(\overline{\mathcal{X}}_i \cap \overline{\mathcal{X}}_j) = \#(\textrm{edges from } v_i \textrm{ to } v_j \textrm{ in } G).$$
Alsno note that $\Delta_{ii} = -\mathrm{deg}(v_i) = \#(\textrm{outgoing edges})$. Hence, $$\Delta = -\mathrm{Deg} + \mathrm{Adj},$$ the so-called \emph{combinatorial Laplacian} of $G$.

The degree map is defined by $$\mathrm{deg} \colon \mathrm{Pic}(\mathcal{X}) \to \mathbb{Z}^{n+1} \colon \mathcal{L} \mapsto (\mathrm{deg}(\mathcal{L}|_{\overline{\mathcal{X}}_0}),\ldots, \mathrm{deg}(\mathcal{L}|_{\overline{\mathcal{X}}_n})).$$
Two different models of $L$, say $\mathcal{L}$ and $\mathcal{L}'$, satisfy $\mathrm{deg}(\mathcal{L}) - \mathrm{deg}(\mathcal{L}) \in \mathrm{Im}(\Delta)$, when we view $\Delta$ as a map $\Delta \colon \mathbb{Z}^{n+1} \to \mathbb{Z}^{n+1}$.

This yields a well-defined map $\tau_{*}$, fitting into a commutative diagram
\[
\xymatrix{
\mathrm{Pic}(X) \ar[r]^{\tau_{*}} \ar[dr]^{\mathrm{deg}} & \mathbb{Z}^{n+1}/\mathrm{Im}(\Delta) \ar[d]^{[(a_0,\ldots,a_n)] \mapsto a_0 + \ldots + a_n} \\
 & \mathbb{Z} 
}
\]
We have $\mathrm{ker}(\mathrm{deg}) = (\mathbb{Z}^{n+1}/\mathrm{Im}(\Delta))_{\mathrm{tors}} =\colon \mathrm{Jac}(G)$ and we know that $\Delta$ has rank $n$.

\begin{exercise}[The Matrix Tree Theorem]
	Show that
	\[
		\#\mathrm{Jac}(G) = \#\{\text{spanning trees of } G\}.
	\]
\end{exercise}

\begin{definition} For the semigroup $\mathrm{Eff}(X) \subseteq \mathrm{Pic}(X)$ generated by $\mathcal{O}(x)$ for $x \in X$, we have $\tau_{*}\mathrm{Eff}(X) \subseteq \mathbb{Z}^{n+1}_{\geq 0} = \colon \mathrm{Eff}(G)$. Then we define $\mathrm{Pic}(G) \colon = (\mathbb{Z}^{n+1}/\mathrm{Im}(\Delta))$.

So $\mathrm{Eff}_1(G) = [v_i],\ldots$, $\mathrm{Eff}_2(G) = [v_i + v_j], \ldots$ and in general, $\mathrm{Eff}_d(G) = \{ [a_0 v_0 + \ldots + a_nv_n] \colon a_i\geq 0, a_0 + \ldots + a_n = d \}$. \end{definition}

\begin{remark} Recall: for $D$ a divisor on $X$, its rank satisfies $r(D) = h^0(\mathcal{O}(D)) - 1 = \mathrm{dim}|D|$. Thus, $r(D)$ is the largest integer $r$ such that for every $E \in \mathrm{Eff}_r(X)_{\overline{\mathbb{C}((t))}}$ we have $[D-E] \in \mathrm{Eff}(X)$. N.B.: For this to hold, we need the ground field to be algebraically closed.\end{remark}

\begin{definition}[Baker-Norine, 2007] For $D = a_0v_0 + \ldots a_n v_n$ a divisor on $G$, $r([D])$ is the largest integer $r$ such that for all $E \in \mathrm{Eff}_r(G)$, $[D-E]$ is contained in $\mathrm{Eff}(G)$. \end{definition}

\begin{remark} If $D$ is not effective, then $r(D) = -1$, and conversely. \end{remark}

\begin{lemma}[Specialisation Lemma, Baker 2008] The rank satisfies $$r(\tau_{*}([D])) \geq r([D])$$ for $D \in \mathrm{Div}(X)$.\end{lemma}
This lemma allows us to "prove theorems on curves by looking at the graphs".

\begin{definition}[S.W. Zhang, 1990] The \emph{canonical divisor on a graph} is given by $$K_G = \sum_v (\mathrm{deg}(v)-2)\cdot v.$$ \end{definition}

\begin{remark} If every component of $\mathcal{X}$ is a $\mathbb{P}^1$, then $K_G = \sum_i \mathrm{deg}(K_{\mathcal{X}}|_{\overline{\mathcal{X}}_i})\cdot \nu_i$. \end{remark}

\begin{theorem}[Tropical Riemann-Roch, Baker-Norine] If $D = a_0 v_0 + \ldots + a_n v_n$ then $$ r([D]) - r([K_G-D]) = \mathrm{deg}(D) + 1 -g,$$ where $g = h^1(G) = e-v+1$ is the first Betti number of the graph, which equals $g(X)$ if $\overline{\mathcal{X}}_1 \cong \mathbb{P}^1$.\end{theorem}

\subsection{Open problems}
\begin{enumerate}
\item Does the Riemann-Roch theorem for curves imply the tropical Riemann-Roch theorem, or conversely?
\item Is there a unified proof of the Riemann-Roch for curves and the tropical Riemann-Roch? Would this be easier?
\end{enumerate}

\subsection{Some remarks}
\begin{enumerate}
\item There are analogues of rank, tropical Riemann-Roch and the specialisation lemma for $g(\mathcal{X}_i) > 0$, by work of Amini and Caporaso (Advances, 2013 or 2014).
\item We could also use metrised complexes (i.e. a simplicial realisation of the special fibre with a metric on the edges), by work of Amini and Baker (to appear in Math. Annalen).
\end{enumerate}

\subsection{Some more exercises}
\begin{exercise} Compute $\mathrm{Jac}(G)$ for $G = \ldots$ 
%TO DO: Include pictures of a triangular graph, a square, and a triangle attached to a square
\end{exercise}
%\begin{solution} $\ZZ/3\ZZ$, $\ZZ/4\ZZ$, $\ZZ/11\ZZ$.
%\end{solution}
\begin{exercise} Show that the map $\mathbb{Z}^{\mathrm{vert}(G)} \to \mathrm{Jac}_{\mathbb{R}}(G)$ vanishes on $\mathrm{Im}(\Delta)$. \end{exercise}
%\begin{solution}
%\end{solution}
